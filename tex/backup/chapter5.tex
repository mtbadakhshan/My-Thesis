% !TeX root=../main.tex
\chapter{کار‌های آینده}
%\thispagestyle{empty} 
\label{chap:results}
\section{مقدمه} 
در روش ارائه شده سعی شده است که بدون نیاز به استفاده از پروتکل و ابزار‌های پیچیده، با حذف فیلتر بلوم و بهره‌گیری از معیار ‌$K$-گمنامی و قرار دادن آدرس‌هایی با احتمال درخواست یکسان، در یک دسته به سطح بالاتری از امنیت نسبت به فیلتر بلوم دست پیدا شود. همچنین در این روش تعیین پهنای باند مصرفی کاملا در اختیار کاربر سبک است و کاربر سبک می‌تواند با سطح امنیت مورد نظرش این مقدار را تعیین کند.

در این پایان‌نامه با فرض این‌که احتمال پرسمان اطلاعات مربوط به هر آدرس $a_n$، به شرطی که مربوط به یک گره کامل نباشد، متناسب است با احتمال استفاده از آن آدرس در شبکه بیت‌کوین به طراحی پرتکل ارائه شده پرداخته شده است. هرچند که این فرض،‌ دور از ذهن نیست اما لازم است که صحت آن از طریق شبیه‌سازی سنجیده شود. به این منظور، در این  پژوهش سعی شد که یک نرم افزار 
\gls{Bitcoin-core}
راه‌اندازی شده تا درخواست‌های گره‌های سبک، که از فیلتر بلوم استفاده می‌کنند، ثبت شود. همچنین سعی شد که با بهره‌گیری از ایدهٔ پایان‌نامهٔ \cite{Nick2015}، که برای کشف آدرس‌های اصلی فیلتر بلوم \lr{PubKey} و \lr{PubKeyHash} آدرس‌های بیت‌کوین را در فیلتر‌های بلوم آزمایش کرده بود، به آدرس‌های اصلی فیلترهای دریافت پی برده شود. در کنار این، یک سرور
الکترام‌ایکس\LTRfootnote{\lr{ElectrumX}}
نیز راه‌اندازی شد تا به صورت مستقیم، بدون استفاده از فیلتر بلوم، نرخ درخواست از طرف کیف‌پول‌های بیت‌کوین الکرترام ثبت شود. اما متاسفانه، به خاطر محدود بودن پهنای باند و الزام استفاده از شبکهٔ 
\gls{Tor}
به خاطر نداشتن آدرس IP استاتیک، فرصت نشد که گره‌ کامل راه‌اندازی شده به صورت کامل به شبکه شناسانده شود و تعداد درخواست‌های قابل توجهی دریافت نماید. از این رو لازم است که برای درستی سنجی فرض انجام شده در آینده شبیه‌سازی‌ گسترده طولانی مدتی انجام گیرد. 

پارامتر $\beta$ در این پروتکل و هچنین مرز بین تکه‌ها در امنیت پروتکل تاثیر مستقیمی دارد. لازم است که در آینده مقدار مشخصی برای این پارامتر محاسبه شود. روش محاسبهٔ امتیازها در این پایان‌نامه یک روش 
\gls{Naive} 
محسوب می‌شود که امکان اجماع همه گره‌ها بر روی امتیاز نهایی را فراهم نموده است. به نظر می‌رسد که با بررسی بیشتر امکان بهره‌گیری از ابزار‌ها و متدهای پیشرفته‌تری برای محاسبهٔ امتیا‌ز‌ها وجود خواهد داشت.

