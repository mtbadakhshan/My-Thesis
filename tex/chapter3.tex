% !TeX root=../main.tex
\chapter{روش تحقیق}
%\thispagestyle{empty} 
\section{مقدمه} 
این فصل، محل شرح کامل روش تحقیق است و بسته به نوع روش تحقیق و با نظر استاد راهنما می‌تواند «مواد و روش‌ها%
\LTRfootnote{Materials and Methods}»
نیز نام بگیرد. این فصل حدود ۱۵ صفحه است.

\section{محتوا (نام‌گذاری بر اساس روش تحقیق و مسأله مورد مطالعه)}
\subsection{علت انتخاب روش}
دلیل یا دلایل انتخاب روش تحقیق را تشریح می‌کند.

\subsection{تشریح کامل روش تحقیق}
برای اینکه پایان‌نامه دارای ارزش علمی باشد، باید قابل تکرار باشد و داوران و خوانندگان از امکان تکرارپذیر بودن کار شما مطمئن شوند. شما باید چگونگی تکرار آزمایش به وسیله دیگران را در این قسمت فراهم کنید. تکرارپذیری آزمایشات و روش شما، برابر با میزان پتانسیل تکرار نتایجِ برابر یا نزدیک به آن است. در زیر به تعدادی از روش‌های تحقیق اشاره شده است:
\begin{itemize}
	\item \textbf{روش تحقیق آزمایشگاهی}\\
	توصیف کامل برنامهٔ آزمایشگاهی شامل مواد مصرفی و نحوهٔ ساخت نمونه‌ها، شرح آزمایش‌ها شامل نحوه تنظیم و آماده‌سازی آزمایش‌ها و دستگاه‌های مورد استفاده، دقت و نحوهٔ کالیبره کردن، شرح دستگاه ساخته شده (در صورت ساخت) و ارائهٔ روش اعتبارسنجی.
	
	\item \textbf{روش تحقیق آماری}\\
	توصیف ابزارهای گردآوری اطلاعات کمی و کیفی، اندازهٔ نمونه‌ها، روش نمونه‌برداری، تشریح مبانی روش آماری و ارائهٔ روش اعتبارسنجی.
	
	\item \textbf{روش تحقیق نرم‌افزارنویسی}\\
	توصیف کامل برنامه‌نویسی، مبانی برنامه و ارائهٔ روش اعتبارسنجی.
	
	\item \textbf{روش تحقیق مطالعهٔ موردی}\\
	توصیف کامل محل و موضوع مطالعه، علت انتخاب مورد و پارامترهایی که تحت ارزیابی قرار داده می‌شوند و ارائهٔ روش اعتبارسنجی.

	\item \textbf{روش تحقیق تحلیلی یا مدل‌سازی}\\
	توصیف کامل مبانی یا اصول تحلیل یا مدل و ارائهٔ روش اعتبارسنجی آن. در ارائه مدل ریاضی معمولاً نیاز است اندیس‌ها، پارامترها، متغیرهای تصمیم و فرمول‌های مدل، به صورت سیستماتیک ارائه شوند. پیشنهاد می‌گردد برای نمایش اندیس‌ها، پارامترها و متغیرهای تصمیم از سه جدول به صورت زیر استفاده گردد:
	\begin{table}[ht]
		\caption{اندیس‌های به کار رفته در مدل ریاضی}
		\label{tab:modelIndices}
		\centering
		\onehalfspacing
		\begin{tabularx}{0.9\textwidth}{|r|X|}
			\hline
			$I, J$	& بیماران \\
			\hline
			$k$		& مرحله زمان‌بندی (بستری، اتاق عمل، ریکاوری) \\
			\hline
			$L_k$	& ماشین (تخت یا اتاق عمل) در مرحله $k$ \\
			\hline
			$n$		&  جراح \\
			\hline
		\end{tabularx}
	\end{table}
	
	\begin{table}[ht]
		\caption{پارامترهای مدل ریاضی}
		\label{tab:modelParameters}
		\centering
		\onehalfspacing
		\begin{tabularx}{0.9\textwidth}{|r|X|}
			\hline
			$t_{ik}$			& زمان خدمت‌دهی به بیمار در مرحله $k$ام \\
			\hline
			$\tilde{t}_{ik}$	& زمان فاری خدمت‌دهی به بیمار در محله $k$ام \\
			\hline
			$t_{ik}^p$			& مقدار بدبینانه (حداکثر) برای زمان خدمت‌دهی به بیمار در مرحله $k$ام \\
			\hline
			$t_{ik}^m$			& محتمل‌ترین مقدار برای زمان خدمت‌دهی به بیمار در مرحله $k$ام \\
			\hline
			$t_{ik}^o$			& مقدار خوشبینانه (حداقل) برای زمان خدمت‌دهی به بیمار در مرحله $k$ام \\
			\hline
		\end{tabularx}
	\end{table}
	
	\begin{table}[ht]
		\caption{متغیرهای مدل ریاضی}
		\label{tab:modelVariables}
		\centering
		\onehalfspacing
		\begin{tabularx}{0.9\textwidth}{|r|X|}
			\hline
			$X_{ild_{k}}$	& متغیر صفر-یک تخصیص بیمار به تخت/اتاق عمل\\
			\hline
			$S_{ild_{k}}$	& زمان شروع خدمت‌دهی به بیمار \\
			\hline
			$Y_{ijkl_{k}}$	& متغیر صفر-یک توالی بیماران \\
			\hline
			$V_{ni}$		& متغیر صفر-یک تخصیص جراح به بیمار‍‍ \\
			\hline
		\end{tabularx}
	\end{table}
	
	\item \textbf{روش تحقیق میدانی}\\
	چگونگی دستیابی به داده‌ها در میدان عمل و نحوه برداشت از پاسخ‌های دریافتی.
\end{itemize}