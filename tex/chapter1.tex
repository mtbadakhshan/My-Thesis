% !TeX root=../main.tex

\chapter{مقدمه}
% دستور زیر باعث عدم‌نمایش شماره صفحه در اولین صفحه‌ی این فصل می‌شود.
%\thispagestyle{empty}
با معرفی زنجیرهٔ بلوکی بیت‌کوین، به عنوان اولین زنجیرهٔ بلوکی، باب تازه‌ای در کاربرد‌هایی که نیاز به اعتماد به یک طرف سوم دارند گشوده شد. زنجیرهٔ بلوکی بیت‌کوین امکان نگهداری و انتقال دارایی را بدون نیاز به اعتماد به هیچ واسطه‌ای، مانند بانک‌ها،‌ فراهم کرد. پیش از آن اعتماد به بانک‌ها دارای ایرادات فراوانی بود که از آن می‌توان به این موارد اشاره کرد: عدم شفافیت، قطع دسترسی افراد به دارایی‌هایشان، کنترل ناعادلانه تورم، نقض حریم خصوصی افراد و غیره. 

رمز ارز بیت‌کوین برای دست‌یابی به چنین امکانی و رفع نواقص بانک‌داری موجود از مجموعه‌ای از مفاهیم و فناوری‌ها مانند یک شبکه همتا‌به‌همتا، پایگاه داده توزیع‌شده‌ای به اسم زنجیرهٔ بلوکی، الگوریتم‌های رمزنگاری جهت صدور و تصدیق تراکنش‌ها و الگوریتم اجماع برای آن‌ که تمام اعضای شبکه بر روی یک زنجیرهٔ بلوکی یکتا توافق داشته باشند. 

در نظام بانکی فعلی موجود حساب افراد مستقل از خود دارایی آن‌ها است ولی در رمز ارز بیت‌کوین هر بیت‌کوین خود دارای ارزش است. به بیان ساده‌تر در نظام بانکی فعلی اگر فردی رمزعبور حسابش را فراموش کند، با احراز هویت حضوری در بانک می‌تواند به رمز عبور جدیدی دسترسی پیدا نماید. همچنین اگر محرز شود که دارایی یک فرد به سرقت رفته است، بانک می‌تواند دارایی فرد متضرر را از سارق پس بگیرد و به حساب اصلی بازگرداند. اما در رمزارز بیت‌کوین، داشتن کلید خصوصی به منزله مالکیت بر دارایی است. اگر کاربر بیت‌کوین کلید خصوصی را گم کند دیگر به دارایی خود دسترسی نخواهد داشت و از طرف دیگر اگر کلید خصوصی وی در دست یک فرد متخاصم قرار بگیرد امکان بازگردانی دارایی وی وجود نخواهد داشت. از این رو امنیت بیت‌کوین دارای چالش‌های بسیار زیاد است.

امنیت در بیت‌کوین از جنبه‌های مختلفی تحلیل می‌شود. در این پژوهش تمرکز اصلی بر روی حریم خصوصی کاربران سبکی است که تمام زنجیره بلوکی را ذخیره نکرده‌اند. عملکرد این کاربران به کاربران دیگری که تمام زنجیره‌ٔ بلوکی را ذخیره کرده‌اند وابسته است. این وابستگی باعث می‌شود که اطلاعات این کاربران نزد کاربری دیگر فاش شود. فاش شدن اطلاعات می‌تواند تبعاتی به همراه داشته باشد. که می‌توان به افشای هویت کاربر سبک در نتجیهٔ آگاهی یک گره دیگر از آدرس‌هایی که مربوط به آن است اشاره نمود. 

اینکه مشخص شود که کدام آدرس‌ها مربوط به کدام کاربر است، می‌تواند باعث فاش شدن تمام فعالیت‌ها و تبادلات مالی آن کاربر شود. علاوه بر این از آن‌جایی که به این طریق می‌توان به دارایی یک فرد پی برد، ممکن است آن فرد در معرض سوء قصد فیزیکی قرار گیرد. چرا که در صورتی که یک فرد بتواند تنها کلید‌های خصوصی قربانی را دریافت نماید، می‌تواند نسبت به تمام دارایی‌های وی در شبکه‌ٔ بیت‌کوین مالکیت داشته باشد. همچنین به خاطر ذات غیر متمرکز بودن این شبکه امکان باز گرداندن دارایی‌های از دست رفته وجود ندارد.

توجه به این نکته ضروری است که با افشای اطلاعات یک کاربر، اطلاعات تمام کاربرانی که با این کاربر مبادله انجام داده‌اند نیز در معرض خطر افشا قرار می‌گیرد. در نتیجه حفظ حریم خصوصی در شبکه‌ٔ بیت‌کوین به جای آن‌که یک امکان باشد،‌ باید به یک الزام تبدیل شود و به صورت ذاتی در پروتکل‌های آن از افشای هویت کاربران جلوگیری شود.

در فصل \ref{def} به تعریف کاربر سبک، بررسی پروتکل ارتباطی وی در شبکهٔ همتا‌به‌همتای بیت‌کوین و مرور آسیب‌پذیری‌های موجود در پروتکل ارتباطی فعلی پرداخته شده است. فصل \ref{LitReview} راه‌حل‌هایی که تاکنون برای حل این مسئله بیان شده‌اند مرور شده است. در فصل \ref{proposed} به بیان راه‌کاری برای جبران نقص پروتکل فعلی پرداخته شده است. در این راه‌حل برخی از ایراداتی که در راه‌حل‌های جایگزین دیگر وجود دارند برطرف شده است.


 


