% !TeX root=../main.tex
% در این فایل، عنوان پایان‌نامه، مشخصات خود، متن تقدیمی‌، ستایش، سپاس‌گزاری و چکیده پایان‌نامه را به فارسی، وارد کنید.
% توجه داشته باشید که جدول حاوی مشخصات پروژه/پایان‌نامه/رساله و همچنین، مشخصات داخل آن، به طور خودکار، درج می‌شود.
%%%%%%%%%%%%%%%%%%%%%%%%%%%%%%%%%%%%
% دانشگاه خود را وارد کنید
\university{دانشگاه تهران}
% پردیس دانشگاهی خود را اگر نیاز است وارد کنید (مثال: فنی، علوم پایه، علوم انسانی و ...)
\college{پردیس دانشکده‌های فنی}
% دانشکده، آموزشکده و یا پژوهشکده  خود را وارد کنید
\faculty{دانشکدهٔ مهندسی برق و کامپیوتر}
% گروه آموزشی خود را وارد کنید (در صورت نیاز)
\department{گروه مخابرات امن و رمزنگاری}
% رشته تحصیلی خود را وارد کنید
\subject{مهندسی برق}
% گرایش خود را وارد کنید
\field{مخابرات امن و رمزنگاری}
% عنوان پایان‌نامه را وارد کنید
\title{تحلیل امنیت یک شبکه همتا‌به‌همتا مبتنی بر زنجیره قالبی}
% نام استاد(ان) راهنما را وارد کنید
\firstsupervisor{دکتر محمّدعلی اخایی}
\firstsupervisorrank{استادیار}
%\secondsupervisor{دکتر راهنمای دوم}
%\secondsupervisorrank{استادیار}
% نام استاد(دان) مشاور را وارد کنید. چنانچه استاد مشاور ندارید، دستورات پایین را غیرفعال کنید.
%\firstadvisor{دکتر مشاور اول}
%\firstadvisorrank{استادیار}
%\secondadvisor{دکتر مشاور دوم}
% نام داوران داخلی و خارجی خود را وارد نمایید.
\internaljudge{دکتر داور داخلی}
\internaljudgerank{دانشیار}
\externaljudge{دکتر داور خارجی}
\externaljudgerank{دانشیار}
\externaljudgeuniversity{دانشگاه داور خارجی}
% نام نماینده کمیته تحصیلات تکمیلی در دانشکده \ گروه
\graduatedeputy{دکتر نماینده}
\graduatedeputyrank{دانشیار}
% نام دانشجو را وارد کنید
\name{محمّدتقی}
% نام خانوادگی دانشجو را وارد کنید
\surname{بدخشان}
% شماره دانشجویی دانشجو را وارد کنید
\studentID{810196369}
% تاریخ پایان‌نامه را وارد کنید
\thesisdate{مهر ۱۳۹۹}
% به صورت پیش‌فرض برای پایان‌نامه‌های کارشناسی تا دکترا به ترتیب از عبارات «پروژه»، «پایان‌نامه» و «رساله» استفاده می‌شود؛ اگر  نمی‌پسندید هر عنوانی را که مایلید در دستور زیر قرار داده و آنرا از حالت توضیح خارج کنید.
%\projectLabel{پایان‌نامه}

% به صورت پیش‌فرض برای عناوین مقاطع تحصیلی کارشناسی تا دکترا به ترتیب از عبارت «کارشناسی»، «کارشناسی ارشد» و «دکتری» استفاده می‌شود؛ اگر نمی‌پسندید هر عنوانی را که مایلید در دستور زیر قرار داده و آنرا از حالت توضیح خارج کنید.
%\degree{}
%%%%%%%%%%%%%%%%%%%%%%%%%%%%%%%%%%%%%%%%%%%%%%%%%%%%
%% پایان‌نامه خود را تقدیم کنید! %%
\dedication
{
{\Large تقدیم به:}\\
\begin{flushleft}{
	\huge
	همسر و فرزندانم\\
	\vspace{7mm}
	و\\
	\vspace{7mm}
	پدر و مادرم
}
\end{flushleft}
}
%% متن قدردانی %%
%% ترجیحا با توجه به ذوق و سلیقه خود متن قدردانی را تغییر دهید.
\acknowledgement{
سپاس خداوندگار حکیم را که با لطف بی‌کران خود، آدمی را به زیور عقل آراست.

در آغاز وظیفه‌  خود  می‌دانم از زحمات بی‌دریغ اساتید  راهنمای خود،  جناب آقای دکتر ... و ...، صمیمانه تشکر و  قدردانی کنم که در طول انجام این پایان‌نامه با نهایت صبوری همواره راهنما و مشوق من بودند و قطعاً بدون راهنمایی‌های ارزنده‌ ایشان، این مجموعه به انجام نمی‌رسید.

از جناب آقای دکتر ... که  زحمت مشاوره‌، بازبینی و تصحیح این پایان‌نامه را تقبل فرمودند کمال امتنان را دارم.

%از همکاری و مساعدت‌های دکتر ... مسئول تحصیلات تکمیلی و سایر کارکنان دانشکده بویژه سرکار خانم ... کمال تشکر را دارم.

با سپاس بی‌دریغ خدمت دوستان گران‌مایه‌ام، خانم‌ها ... و آقایان ... در آزمایشگاه ...، که با همفکری مرا صمیمانه و مشفقانه یاری داده‌اند.

و در پایان، بوسه می‌زنم بر دستان خداوندگاران مهر و مهربانی، پدر و مادر عزیزم و بعد از خدا، ستایش می‌کنم وجود مقدس‌شان را و تشکر می‌کنم از خانواده عزیزم به پاس عاطفه سرشار و گرمای امیدبخش وجودشان، که بهترین پشتیبان من بودند.
}
%%%%%%%%%%%%%%%%%%%%%%%%%%%%%%%%%%%%
%چکیده پایان‌نامه را وارد کنید
\fa-abstract{
کاربران سبک، بخش قابل توجهی از شبکه‌ٔ همتا‌به‌همتای بیت‌کوین را تشکیل می‌دهند. از طرف دیگر این کاربران برای دریافت اطلاعات خود به گره کامل وابسته هستند، در نتیجه بخش قابل توجهی از اطلاعات آن‌ها نزد گره‌های کامل فاش می‌شود. از این رو حفظ حریم خصوصی آن‌ها دارای اهمیت زیادی است. استفاده از فیلتر بلوم که در طرح $37$م پیشنهاد بهبود بیت‌کوین، به عنوان اولین راه‌حل حفظ حریم خصوصی کاربران سبک مطرح شد، دارای ایرادات اساسی بسیار زیادی است. در این روش از یک سری آدرس‌های پوششی ، در نتیجهٔ خطای نوع دو فیلتر بلوم، برای پنهان کردن آدرس کاربر سبک استفاده شده است. در این پایان‌نامه ضعف‌های استفاده از فیلتر بلوم در شبکهٔ همتا‌به‌همتای بیت‌کوین بیان شده است. همچنین مروری بر راه‌حل‌هایی که برای به عنوان جایگزین فیلتر بلوم یا بهبود دهنده‌ آن مطرخ شده‌اند انجام شده است.  

در این پایان‌نامه سعی شده است راه‌حلی ارائه شود که برخلاف فیلتر بلوم که آدرس‌های پوششی به صورت شانسی و کورکورانه انتخاب می‌شدند، به صورت هوشمندانه این آدرس‌ها انتخاب شودند. آدرس‌های انتخابی در این روش به نحوی انتخاب شده‌اند که آنتروپی مجموعه آدرس‌های درخواست شده بیشینه شده و گره کامل با ابهام بیشتری مواجه شود. همچنین در این روش امکان تنظیم پهنای باند مصرفی به ازای هر درخواست برای کاربر سبک فراهم شده است. 
}
% کلمات کلیدی پایان‌نامه را وارد کنید
\keywords{زنجیرهٔ بلوکی، بیت‌کوین، حفظ حریم خصوصی، کاربر سبک، درستی سنجی پرداخت ساده‌شده}
% انتهای وارد کردن فیلد‌ها
%%%%%%%%%%%%%%%%%%%%%%%%%%%%%%%%%%%%%%%%%%%%%%%%%%%%%%
